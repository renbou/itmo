\documentclass[times]{itmo-student-thesis}

%% Список источников в отдельном файле.
\usepackage{filecontents}
\begin{filecontents}{bachelor-thesis.bib}
@book{ bellman,
    author      = {R. E. Bellman},
    title       = {Dynamic Programming},
    address     = {Princeton, NJ},
    publisher   = {Princeton University Press},
    numpages    = {342},
    pagetotal   = {342},
    year        = {1957},
    langid      = {english}
}
\end{filecontents}
\addbibresource{bachelor-thesis.bib}

\begin{document}

%% \publishyear нужен для \printmainbibliography, которая далее выводит список источников по годам. 
\publishyear{2024}
%% \maketitle отредактирован, чтобы не выводить титульный лист и остальные страницы, т.к. они сейчас не нужны.
\maketitle{Бакалавр}

%%%%
%% Оглавление
%%%%
\tableofcontents

%%%%
%% Введение
%%%%
\startprefacepage

В данном разделе размещается введение.

%%%%
%% Первая глава
%%%%
\chapter{Первая глава}

%% Пример ссылок.
Вне обзора:~\cite{bellman}.

\section{Листинги}

В работах студентов кафедры <<Компьютерные технологии>> часто встречаются листинги. Листинги бывают
двух основных видов~--- исходный код и псевдокод. Первый оформляется с помощью окружения \texttt{lstlisting}
из пакета \texttt{listings}, который уже включается в стилевике и немного настроен. Пример Hello World на Java
приведен на листинге~\ref{lst1}.

\begin{lstlisting}[float=!h,caption={Пример исходного кода на Java},label={lst1}]
public class HelloWorld {
    public static void main(String[] args) {
        System.out.println("Hello, world!");
    }
}
\end{lstlisting}

%%%%
%% Заключение
%%%%
\startconclusionpage

В данном разделе размещается заключение.

\printmainbibliography

%% После этой команды chapter будет генерировать приложения, нумерованные русскими буквами.
%% \startappendices из старого стилевика будет делать то же самое
\appendix

\end{document}
